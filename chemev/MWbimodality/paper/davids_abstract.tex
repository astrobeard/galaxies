We develop a hybrid model of galactic chemical evolution that combines a
multi-ring computation of chemical enrichment with the predictions of a 
cosmological hydrodynamic disk galaxy simulation for the radial migration and
vertical distribution of stellar populations.  Our fiducial enrichment model 
adopts empirically motivated forms of the star formation law and star formation 
history, and a gradient in outflow mass-loading $\eta(\Rgal)$ tuned to 
reproduce the observed metallicity gradient.  With the simulation-based
migration recipe, this model reproduces many of the striking qualitative 
features of the Milky Way disk's abundance structure: (i) the dependence
of the [O/Fe]-[Fe/H] distribution on $\Rgal$ and midplane distance $|z|$;
(ii) the changing shapes of the [O/H] and [Fe/H] distributions with $\Rgal$
and $|z|$; (iii) a broad distribution of [O/Fe] at sub-solar metallicity and
shifts of the [O/Fe] distribution with $\Rgal$, $|z|$, and [Fe/H]; (iv) a
tight correlation between [O/Fe] and stellar age for [O/Fe]$>0.1$; (v) a
population of young and intermediate-age $\alpha$-enhanced stars caused by
migration-induced fluctuations in the Type Ia supernova rate; 
(vi) non-monotonic [O/H]-age and [Fe/H]-age relations, with large scatter
and a median age $\sim 4\Gyr$ near solar metallicity.  Observationally 
motivated models with a burst of star formation $\sim 2\Gyr$ ago produce
better agreement with the observed [Fe/H]-age relation but worse agreement 
with the observed [O/Fe]-age relation.  None of our models produces an [O/Fe] 
distribution with the distinct bimodality seen in the observations, which 
suggests that more discontinuous forms of the early star formation history 
are required in addition to radial migration.  All code and tables used for 
our models are publicly available through {\tt github.com/giganano/VICE}.
