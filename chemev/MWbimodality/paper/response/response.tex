
\documentclass{report} 
\usepackage[margin = 0.6in]{geometry} 
\usepackage{amssymb} 
\usepackage{amsmath}
% \linespread{1.8} 

\newcommand{\breakline}{\noindent\rule{\textwidth}{1pt}} 

\begin{document} 
Dear editor, 
\par\null\par 
We thank our referee, Dr. J. Ted Mackereth, for his thorough comments on our 
submission. 
We greatly appreciate his insight, and we believe our manuscript 
has improved as a result of his efforts. 
\par
Our responses are below, interleaved with comments from Dr. Mackereth in 
bold. 
Substantial changes to the revised manuscript are marked in red. 

\par\null\par 
\breakline 
\par\null\par 
\textbf{
	The paper is well put together, if quite long and detailed! 
	Much of the latter portion of the paper is given over to detailed 
	descriptions of comparisons between the model predictions and the APOGEE 
	data. 
	The authors are refreshingly upfront about places where their model fails 
	to correctly predict observations. 
	However, I felt in general the authors might have more strongly emphasised 
	what new predictions the models provide for the Milky Way's formation and 
	evolution. 
	As an example, more than 50\% of the abstract is a list of agreement rather 
	than any novel predictions! 
	This is subjective, of course, so I am not explicitly asking for changes 
	here.
} 
\par 
Having strived for transparency, we are glad that Dr. Mackereth believes that 
we achieved this goal and that it improves the quality of the paper! 
Although novel predictions are always interesting, we believe that one of the 
most important results from this paper is that some observables can be 
explained by ongoing star formation combined with stellar migration, but that 
simultaneously fitting all of them is extremely challenging. 
None of the models we present are able to achieve this goal despite how well 
one model may agree with a particular observable. 
Largely because we find that different observables from the same dataset favor 
different models, an important lesson to learn from this paper is that we as a 
community need to be careful to compare our models to as many observables in as 
many Galactic regions as possible. 
This was our motivation behind emphasizing the comparison with data. 

\par\null\par 
\breakline 
\par\null\par 
\textbf{
	Does the imposed SFH match in any quantitative way that from the 
	simulation? 
	I was concerned that this model imposes some star formation beyond R $>$ 
	13kpc at times as early as 11-12 Gyr ago. 
	While small, I would presume that these annuli try to match at least a few 
	star particles from the simulation at these times. 
	Given the birth radii distribution of the simulations at 8-10Gyr old, then 
	I would expect that the model tags an 'analogue' at a very different radius 
	to the implied birth radius (given the current description of the tagging 
	process). 
	Is this the case? 
	If not, how is this handled? 
	If so, how important is this for the radial migration prescription? 
	Some text describes this, but I felt that the paper would benefit from some 
	more detail here. 
	I would imagine that the dynamics of stars forming at R $<$ 5kpc are very 
	different to those forming at large radii at early times. 
} 
\par 
The imposed SFH is not taken from the simulation. 
Instead, it is intended to resemble a simple assumption that one might take for 
a galaxy like the Milky Way, where ``simple'' for the sake of chemical 
evolution models often means neglecting effects like the radial growth of the 
Galaxy. 
Dr. Mackereth's understanding of the tagging process is accurate; the model 
indeed matches these stellar populations born at large radii and early times 
with star particles from the h277 simulation, but the number of star particles 
in our sample is sufficiently large that this is not a cause for concern. 
For each of the stellar population and star particle pairs in our fiducial 
model, we have recorded the difference in birth radius between the two and 
analyzed the resulting distributions. 
We find that even at large radii (R $\gtrsim$ 10 kpc) and early times (age $>$ 
10 Gyr), the majority of stellar populations are assigned an analogue that 
formed within~$\sim$2 kpc of its birth radius. 
Stellar populations born at, e.g., R = 13 kpc are in general not being tagged 
with analogues that formed as far in as R = 5 kpc, instead finding star 
particles that formed at~$\sim$11 kpc if the search widens that far without 
finding another analogue first. 
We have added a few sentences to this effect to paragraph 4 of~\S~2.2 so 
that uninitiated readers can be assured that our models assign outer rather 
than inner disc star particles to these outer disc stellar populations. 
\par 
This issue is of little importance to the conclusions presented in this paper. 
In our exploratory work, we originally used an algorithm in which stellar 
populations that did not find an analogue within~$R~\pm$~500 pc and~$T~\pm$~500 
Myr simply stayed at their radius of birth. 
These models make similar predictions to the ones presented in the paper, which 
we take as an indication that the fine details of the dynamical history do 
not impact our results. 
In general, the conclusions we discuss are impacted more so by the 
fact~\textit{that} stars are migrating and not so much by~\textit{how} stars 
are migrating. We have also added two sentences to this effect to paragraph 4 
of~\S~2.2. 

\par\null\par 
\breakline 
\par\null\par 
\textbf{
	Similarly to above, my reading of the current procedure is that it may 
	imply some biases as to where the analogues are selected. 
	Since the search is conducted in cylindrical coordinates, then surely the 
	search volume will be vastly different dependent on the radii at which it 
	is conducted (since delta R is constant)? 
	i.e at R = 1kpc, a~$\pm$250pc annulus is~$\sim$10 times smaller than one at 
	15kpc. 
	Perhaps this has no effect, but to the uninitiated reader this seemed like 
	it could be a concern.
} 
\par 
It is true that the search annuli increase in size with increasing radius, but 
we do not believe that this has any effect on the model. 
This point is actually closely connected to the previous question; the larger 
search annuli at large~$R$ help ensure that the stellar populations born there 
are able to find an analogue born at a similar radius without having to extend 
the search too far inward. 
A change to our model such that stars stay at their birth radius if they do not 
find an analogue within~$R~\pm$~500 pc and~$T~\pm$~500 Myr makes similar 
predictions (see previous). 
Particularly at large~$R$, this is a substantial change to the dynamics of the 
stellar populations such that the search may as well have not even been 
conducted for some of them, and our conclusions are still unaffected. 
We hope that the aforementioned addition of text to paragraph 4 of~\S~2.2 
helps mitigate this concern. 
There may be interesting regions of parameter space where the model predictions 
are sensitive to the fine details of the dynamical history and decisions like 
this would matter, but in practice we find that the models in the present paper 
are unaffected. 

\par\null\par 
\breakline 
\par\null\par 
\textbf{
	Finally, the authors use the $z$ dependence in their comparisons but (as 
	far as I could see) did not mention how this was handled. 
	I assumed that the final $z$ of the simulation particles is adopted, but it 
	would be good to state this somewhere explicitly. 
} 
\par 
We thank Dr. Mackereth for letting us know that this was not made clear enough. 
His assumption was correct; it is indeed taken from the h277 analogue assigned 
to a stellar population. 
To address this, we have modified sentence 5 in paragraph 3 of~\S~2.2 to say 
that the stellar population adopts the present-day midplane distance~$z$ and 
the change in radius~$\Delta R_\text{gal}$ from its analogue. 
This sentence previously stated this only for the change in radius. 
We have also added a table of our model parameters near the end of~\S~2, and 
it is stated there that both~$\Delta R_\text{gal}$ and~$z$ are taken from 
the h277 analogue. 

\par\null\par 
\breakline 
\par\null\par 
\textbf{
	I think it would be good to explicitly mention that h277 has~$\sim$the same 
	final scale length as the Milky Way, I had to go digging for that 
	information when I was thinking about the above points.
} 
\par 
We have added this statement near the end of paragraph 3 of~\S~2.2. 

\par\null\par 
\breakline 
\par\null\par 
\textbf{
	The 'inside-out' SFH implies that star formation peaked at 10-11 Gyr ago in 
	the very outer disc - is this really a good assumption? 
}
\par 
For reference, we are duplicating equation 10 in section 2.5 here, which 
describes the time-dependence of the SFH in the fiducial model: 
\begin{equation} 
f_\text{IO}(t|R_\text{gal}) = (1 - e^{-t/\tau_\text{rise}})
e^{-t/\tau_\text{sfh}}. 
\end{equation} 
As mentioned in the text, this equation does have a peak 
near~$\tau_\text{rise}$, but in detail the time of the peak increases with 
increasing~$\tau_\text{sfh}$ (i.e. with increasing~$R_\text{gal}$). 
When~$\tau_\text{sfh}$ is as high as~$\sim$30 Gyr, the peak in star formation 
instead occurs~$\sim$7.5 - 8 Gyr ago. 
Whether or not this is accurate for the Milky Way, it at least to some extent 
reflects the radial growth of the Galaxy. 
We have added a couple of sentences to the discussion of equation 10 in~\S~2.5 
to clarify this. 

\par\null\par 
\breakline 
\par\null\par 
\textbf{
	An assumption is made that suggests radial migration does not alter the 
	surface density profile of the Galaxy. 
	Is this assumption backed up by the simulation that is used? 
} 
\par 
Our star particle sample from h277 indeed supports this assumption. 
We have added such a statement to the first paragraph of~\S~2.5 and the third 
paragraph of~\S~2.7. 

\par\null\par 
\breakline 
\par\null\par 
\textbf{
	A tabulated version of the free parameters of the model would be a useful 
	reference, as I found myself flicking back and forth quite a lot while 
	reading the paper. 
} 
\par 
We have constructed such a table, and placed it near the and of our methods 
section. 
It is referenced in the text at the beginning of~\S~2 (in the final sentence 
before~\S~2.1) and again at the beginning of~\S~2.8. 

\par\null\par 
\breakline 
\par\null\par 
\textbf{
	How does the inclusion of bulge-like particles from the simulation change 
	things in the inner disk? 
} 
\par 
Although they are included in VICE's public code base, bulge-like particles are 
not included in the sample used in the models for this paper. 
We state this in the final two sentences of paragraph 4 in~\S~2.1. 
As a result, every star particle assigned as an analogue is one with disc-like 
kinematics. 
We have added a statement to paragraph 3 of~\S~2.2 to clarify this at the same 
time the analogue search process is described. 

\par\null\par 
\breakline 
\par\null\par 
\textbf{ 
	The paper discusses intermediate-age alpha-enhanced stars at length but 
	does not provide an idea of their [Fe/H], which would of course provide a 
	more detailed constraint as to their origin in the model. 
} 
\par 
This is an excellent suggestion for further discussion on this prediction of 
our model. 
We have added the relevant statements to the end of paragraph 5 of~\S~3.4. 

\par\null\par 
\breakline 
\par\null\par 
\textbf{
	Fig 10: I wondered how much the noted difference between the model and the 
	observations at 3-5kpc might be due to the presence of low metallicity bar 
	stars at this annulus in the data (for example those noted in Bovy et al. 
	2019). 
	A flattening of the surface density profile implied in the central region 
	may make differences here too, I think?
} 
\par 
This is an excellent point! 
We have added a reference to the metallicity map published in Bovy et al. 
(2019) to paragraph 3 of~\S~3.2. 
However, these models are generally independent of the mass normalization - it 
matters in determining how far up the~$\Sigma_\text{gas}-\dot{\Sigma}_\star$ 
relation our models go, but this has only a small effect on chemistry. 
For this reason, we are skeptical that changing the surface density profile 
would significantly impact the MDFs, but we have not ran models with different 
choices for the surface density gradient, so we cannot say for certain. 
We refrain from mentioning anything about surface density in our additions 
to the text. 

\par\null\par 
\breakline 
\par\null\par 
\textbf{ 
	Sec 3.4, paragraph 4: The description of the SN Ia rate dropping causing an 
	increase in [O/Fe] is confusing. 
	Especially since gas is not mixing between annuli, I would be surprised if 
	CCSNe could raise the [O/Fe] again once any SNIa had occurred? 
	(I think this implies that at some annuli the radial migration is extremely 
	rapid/efficient and mixing stars out of the annulus on timescales far less 
	than the SNIa timescale, right?) 
} 
\par 
We thank Dr. Mackereth for pointing out that this is confusing. 
It is true that at some annuli radial migration is efficient and mixes stars 
on timescales shorter than or comparable to the SN Ia timescale, but that's 
only one piece of this puzzle. 
There's also migration on longer timescales coupled with the long tail of the 
SN Ia DTD - our DTD is nearly a~$t^{-1}$ power-law, and in that case only half 
of the SNe Ia explode between 100 Myr and 1 Gyr, the other half occurring 
between 1 Gyr and 10 Gyr, leaving plenty of time for radial migration. 
\par 
It is true that [O/Fe] can increase again once SNe Ia have started to go 
off; indeed this is what happens during a starburst event (for details, see the 
Johnson \& Weinberg 2020 paper). 
The reason for this is that it's not the supernovae since~$t = 0$ that 
establish the current chemical composition of the ISM, but rather those that 
occurred approximately within the previous depletion time (see the Weinberg, 
Andrews \& Freudenburg 2017 paper). 
In our models, the depletion time is short at large radii due to the 
substantial mass loading factors there, so the ISM responds quickly to 
perturbations in the core-collapse to type Ia supernova ratio. 
We have expanded our discussion in paragraph 4 of~\S~3.4 in order to 
communicate this more clearly. 

\par\null\par 
\breakline 
\par\null\par 
\textbf{
	Sec 3.4: I think the Miglio+ 2020 paper suggests that at least all the 
	Kepler over-massive alpha-rich stars are all explained by mass-transfer, 
	but I could be wrong. 
	I think the age-difference scale between this study and the Warfield+ paper 
	are also significantly different, and likely are due to different 
	phenomena, but I may be wrong! 
} 
\par 
Our understanding of Miglio et al. (2021)'s argument is that at least a 
substantial fraction of the young~$\alpha$-rich population can be attributed 
to mass transfer, but a comparison to binary population synthesis models is 
necessary to make stronger claims. 
The Hekker \& Johnson (2019) paper argues that a portion of the 
young~$\alpha$-rich stars seen in APOGEE are intrinsically young, because 
otherwise their data would make sense only if they're the result of mergers on 
the main sequence. 
These two papers are consistent with one another if a smaller but still 
statistically significant portion of the young~$\alpha$-rich population is made 
of truly young stars not formed via mass transfer. 
Our interpretation isn't mutually exclusive with the mass transfer scenario, 
because they would simply explain different sub-populations of 
young~$\alpha$-rich stars - the model successfully reproduces those which are 
intrinsically young, but we do not correct our abundances for any binary star 
interactions. 
We have added additional discussion to paragraph 5 of~\S~3.4 to make this 
distinction a little clearer. 

\par\null\par 
\breakline 
\par\null\par 
\textbf{
	I was confused as to why the 'Late-burst' model was described as having a 
	better agreement to Feuillet et al than the Inside-Out. 
	To an untrained eye they appear to both match the data in some regions 
	better than the other. 
	I think this is further evidence that a more definite bimodality might be 
	achieved by a double burst SFH? 
} 
\par 
We thank Dr. Mackereth for letting us know that this explanation is confusing. 
In general, the inside-out model overpredicts ages for solar and sub-solar 
metallicity stars at~$R = 5 - 7$ kpc, an issue which is mitigated by the 
late-burst model. 
In the outer Galaxy, the late-burst model better reproduces the C-shaped nature 
of the AMR as observed by Feuillet et al. (2019). 
Although there appears to be a difference in normalization at large~$R$ in the 
late-burst model, this could be due to a number of minor details: a 
mis-calibration of our yields or the mass-loading factor, slight differences in 
the time-dependence of the burst as a function of radius, etc. 
We have revised our discussion in~\S~3.5 to address this. 
Perhaps this difference is shown more clearly in the age-[O/H] relation, which 
is unaffected by SN Ia enrichment; this is shown in Fig. 18. 

\par\null\par 
\breakline 
\par\null\par 
\textbf{
	I believe that you meant to cite Mackereth et al (2018) in the introduction 
	when referring to the bimodality in simulations. The Mackereth et al (2017) 
	paper which is cited currently concerns modelling of the stellar surface 
	density (and may be of relevance to your discussion of that!). 
} 
\par 
We thank Dr. Mackereth for pointing this out! 
We have updated the draft to cite his 2018 paper. 

\par\null\par 
\breakline 
\par\null\par 
\textbf{
	Further, the results and discussion in Mackereth et al (2018) may be 
	relevant to your discussion at the tail-end of section 3.3 - since we also 
	concluded that a hybrid two-infall + migration is needed for bimodality, 
	and we suggested the cause of this in the Milky Way was an atypical early 
	mass assembly. 
} 
\par 
We have added a sentence referencing this. 

\par\null\par 
\breakline 
\par\null\par 
\textbf{
	A couple of typos I spotted along the way: 
	\begin{itemize} 
    \item{
	    intro: Rejuvinate~$\rightarrow$ rejuvenate
    }
    \item{
    	Sec 3.3, para 2: I think the bins are 0.04 dex, not 0.4? 
    }
	\end{itemize} 
}
\par 
We thank Dr. Mackereth for catching these editing mistakes! 


\end{document} 

