\newcommand{\afe}{[\alpha/{\rm Fe}]}
\newcommand{\ofe}{[{\rm O}/{\rm Fe}]}
\newcommand{\feh}{[{\rm Fe}/{\rm H}]}
\newcommand{\oh}{[{\rm O}/{\rm H}]}
\newcommand{\h277}{{\tt h277}}

The large dispersion in the age-metallicity relation of local disk stars and
the high metallicity of the sun relative to nearby stars of similar age
provided early evidence that stars in the Milky Way disc can migrate many
kpc from the Galactocentric radius at which they formed
\citep{Edvardsson1993,Wielen1996}.
Interest in radial migration as an important element of galactic chemical
evolution (GCE) grew further with the demonstration by \cite{Sellwood2002}
that resonant interactions with transient spiral perturbations could change
stars' orbital guiding center radii without increasing orbital eccentricity,
and with subsequent studies showing ubiquitous radial migration in 
numerical simulations of disc galaxies (e.g., 
\citealt{Roskar2008a,Roskar2008b,Loebman2011,Bird2012,Bird2013,Grand2012,
Kubryk2013}).
% This list could get much longer if we went past 2013, but given the
% historical nature of the paragraph I think it is reasonable to focus
% on the earliest examples.  However, my list there might still be incomplete.
\cite{Schoenrich2009a,Schoenrich2009b} developed the first detailed GCE
models incorporating radial migration, describing it with a flexible
dynamically motivated parameterization constrained simultaneously with
other GCE parameters when fitting to observations.  A number of subsequent
studies have incorporated radial migration using similar analytic or
parameterized models (e.g.,
\citealt{Bilitewski2012,Hayden2015,Kubryk2015a,Kubryk2015b,Feuillet2018,
Sharma2020}), and
\cite{Frankel2018,Frankel2019,Frankel2020} have used stellar abundances,
ages, and kinematics to constrain radial migration empirically.

In this paper we construct evolutionary models for the Milky Way disc that
combine a classic multi-ring GCE approach (e.g.,
\citealt{Matteucci1989,Wyse1989,Prantzos1995})
with the stellar migration predicted by a hydrodynamic simulation of disc
galaxy formation from cosmological initial conditions.  Our methodology
is similar to that of \cite{Minchev2013,Minchev2014} and has similar 
motivations.  The use of a cosmological simulation that agrees with
many observed properties of the Milky Way assures that our stellar 
migration scenario is physically plausible, including any correlations 
of migration in time and space that might be difficult to capture in a
parameterized description.  Most hydrodynamic cosmological simulations include
metal enrichment, and direct comparison between the predicted and observed
abundance patterns can provide valuable insights into the accuracy of the
simulations and the possible origin of the observed element structure (e.g., 
\citealt{Mackereth2017,Grand2018,Buck2020,Vincenzo2020}.
% The reference list for this sentence is potentially semi-infinite.
% As a motivated way to make it finite, I restricted to cases using fully
% cosmological initial conditions (zoom in or large box) that go into some
% detail with, e.g., distributions of abundance ratios in stars, not just
% overall metallicity and gradients.  Within these restrictions there still
% may be multiple papers to add.
However, many ingredients of the simulations' enrichment recipes are uncertain,
and metal transport and mixing within the interstellar medium (ISM) are
sensitive to numerical resolution and to details of the hydrodynamics and
star formation algorithms.  Our hybrid approach allows us to consider many
choices for uncertain GCE parameters, tuning them to reproduce some 
observations while leaving others as independent empirical tests.
This flexible approach also allows us to isolate the impact of different
GCE model ingredients and to zero-in on the ways that stellar migration
influences the predicted chemical evolution.  
The hybrid model is not fully self-consistent, since it adopts its own 
accretion, star formation, and outflow histories rather than the simulation's.
In this paper, furthermore, we use a single cosmological simulation and 
therefore a single realization of migration history, though one could
apply the same method to multiple simulations to predict a statistical
distribution of outcomes.

We focus our predictions and observational comparisons on oxygen, a 
representative $\alpha$-element produced almost exclusively by core
collapse supernovae (CCSN), and iron, which at solar abundances has
roughly equal contributions from CCSN and Type Ia supernovae (SNIa).
We will consider other elements with other nucleosynthetic sources in
future work, but observed trends of $\feh$ and 
$\afe$\footnote{We follow standard
notation $[{\rm X}/{\rm Y}] = \log_{10}[(X/Y)\div (X/Y)_\odot]$.  Different
observational studies use different $\alpha$-elements (or combinations
thereof) in abundance ratios, and we will generally use $\ofe$ and $\afe$
synonymously.} 
in the Milky Way disk already display a number of striking features, including:
\begin{itemize}
\item At sub-solar $\feh$, the distribution of $\afe$ is bimodal, with a
  ``high-$\alpha$'' sequence and ``low-$\alpha$ sequence'' typically
  separated by 0.1-0.4 dex (e.g., 
  \citealt{Furhmann1998,Bensby2003,Adibekyan2012,Vincenzo2021}).
\item The location of the high-$\alpha$ and low-$\alpha$ sequences is nearly
  independent of position in the disc, but the relative number of stars
  in these sequences and the distribution of those stars in $\feh$ changes
  systematically with Galactocentric radius $\Rgal$ and midplane distance $|z|$
  \citep{Nidever2014,Hayden2015,Weinberg2019}.
\item In addition to an overall radial gradient, the shape of the $\feh$
  distribution for stars with $|z|<0.5\kpc$ changes from negatively skewed
  in the inner disc to roughly symmetric at the solar neighbourhood to
  positively skewed in the outer disc \citep{Hayden2015,Weinberg2019}.
\item With increasing $|z|$, $\feh$ distributions become more symmetric
  and less dependent on $\Rgal$ \citep{Hayden2015}.
\item The age-metallicity relation (AMR) is broad, with a wide range of
  $\feh$ at fixed stellar age and vice versa, in the solar neighbourhood
  \citep{Edvardsson1993} and beyond \citep{Feuillet2019}.  The trend of
  median age with $\feh$ or $\oh$ is non-monotonic, with solar metallicity
  stars being younger on average than both metal-poor {\it and} metal-rich
  stars \citep{Feuillet2018,Feuillet2019}.
\item The trend of stellar age with $\afe$ is much tighter than the trend
  with $\feh$, becoming broad near $\afe \approx 0$ 
  \citep{Feuillet2018,Feuillet2019}.  Although most stars with $\afe \geq 0.1$
  are old, observations have revealed a significant population of
  $\alpha$-rich stars that appear to be young or intermediate age
  \citep{Chiappini2015,Martig2015,Warfield2021}.  Some of these stars
  may have been ``rejuvenated'' by stellar mergers or mass transfer 
  \citep{Izzard2018}, and the question of what fraction are truly much
  younger than the median age-$\afe$ relation remains open \citep{Hekker2019}.
\end{itemize}
Many of these results have emerged most clearly from the Apache Point 
Observatory Galactic Evolution Experiment (APOGEE; \citealt{Majewski2017})
of the Sloan Digital Sky Survey (SDSS-III, \citealt{Eisenstein2011};
SDSS-IV, \citealt{Blanton2017}), sometimes confirming and extending trends
suggested by earlier observational data.  We will assess the degree to 
which models with fairly conventional GCE assumptions coupled to 
simulation-based radial and vertical migration of stars can explain, or
fail to explain, these observations.

Relative to \cite{Minchev2013,Minchev2014}, our base GCE model has many 
differences of detail, the most important being our inclusion of outflows,
which is in turn connected to our different choice of oxygen and iron yields.
Our simulation, the galaxy \h277\ from the \cite{Christensen2012} suite
evolved with the N-body+SPH code {\sc GASOLINE}	\citep{Wadsley2004}, is
fully cosmological, while the simulation used by \cite{Minchev2013,Minchev2014}
has a more idealized geometry with merger and accretion history drawn from
a larger cosmological volume \citep{Martig2012}.  
The \cite{Minchev2013,Minchev2014} simulation has a fairly strong, long-lived
bar while \h277\ has only a weak, transient bar, and this difference could
have some impact on radial migration.  Another methodological difference, 
which turns out to be important for some observables, is that we track 
enrichment from stellar populations as they migrate (see \S\ref{sec:2p2} below),
while \cite{Minchev2013,Minchev2014} assume that populations enrich only
the radial zone in which they were born.

Previous studies have shown that \h277\ and other disc galaxies evolved with
similar physics have realistic rotation curves 
\citep{Governato2012,Christensen2014},
stellar mass \citep{Munshi2013}, metallicity \citep{Christensen2016},
dwarf satellite populations \citep{Zolotov2012,Brooks2014}, and 
HI properties \citep{Brooks2017}.  
% List cribbed from Jon's paper; I didn't re-examine.
Most directly relevant to this study,
\cite{Bird2020} show that \h277\ accurately reproduces the observed relation
between stellar age and vertical velocity dispersion $\sigma_z$.
This relation arises in part because of ``upside down'' disc formation in
which the star-forming gas layer becomes thinner with time
\citep{Bournaud2009,Forbes2012,Bird2013} and in part because stars are
dynamically heated over time.
\cite{Schoenrich2009a} distinguish between the radial mixing caused by
``blurring'' of stars on moderately eccentric orbits and ``churning'' that
changes the guiding center radii of their orbits.  Both phenomena occur in
our simulation and we do not attempt to separate them, using the terms
``radial migration'' or ``radial mixing'' to cover their combined effect.
In addition to radial migration, we use the \h277\ predictions for the
vertical locations (i.e., midplane distances $|z|$) of stars at the 
present day.  Our GCE model assumes that the gas disk is vertically well
mixed, so a stellar population's birth abundances depend only on $\Rgal$
and time.  Vertical gradients arise because older populations have
larger $\sigma_z$ and thus larger average $|z|$, and also because radial
migration is coupled to changes in $\sigma_z$ \citep{Solway2012}.
%% JON: OTHER PAPERS WE SHOULD CITE HERE?
The good match to the observed age-velocity relation found by \cite{Bird2020}
allows us to use vertical trends of abundance distributions as a further
test of our chemical evolution model.

We describe the \h277\ simulation further in \S\ref{sec:2p1} and our
implementation of radial migration in \S\ref{sec:2p2}.
We describe the base GCE model in \S\S\ref{sec:2p3}-\ref{sec:2p6}.
Distinctive features of our GCE model are the use of radially dependent
mass outflow loading $\eta(\Rgal)$ to tune the metallicity gradient,
our implementation of a star formation law motivated by spatially resolved
studies of nearby galaxies and high-redshift studies of its time dependence,
and our use of mean radial age trends of disc galaxies to set the radial
dependence of the star formation history (SFH).  Our fiducial model adopts
a smooth SFH with an ``inside out'' radial trend in which star formation
proceeds more rapidly in the inner Galaxy.  Motivated by the observational
analysis of \cite{Isern2019} and \cite{Mor2019}, we also consider models with
a burst of star formation centred $\approx 2\Gyr$ in the past, similar to 
the one-zone models investigated by \cite{Johnson2020}.  Other authors have
suggested multiple bursts in the Milky Way's SFH 
(e.g., \citealt{Lian2020a,Lian2020b,Sysoliatina2021}, OTHERS?),
perhaps triggered by satellite interactions, while others have advocated
a two-phase star formation history to explain the $\afe$ dichotomy
(e.g., \citealt{Chiappini1997,Haywood2016,Spitoni2019,Buck2020,Khoperskov2021}).
% I don't think this is the place to reference every paper using the 2-infall
% model, but we could add one or two more references if they clearly belong
% with this sentence, especially if they bring in a distinct group.
We do not investigate these more complex SFHs here, but we plan to do so
in future work.
