
We have compared our fiducial inside-out SFH model and its variants to a 
variety of observations, most of them derived from the SDSS APOGEE survey
\citep{Majewski2017}, finding a number of qualitative successes but also
some significant quantitative discrepancies.

\begin{enumerate}

\item The relative number of high-$\alpha$ and low-$\alpha$ stars and the
  $\feh$ distribution of these two populations changes systematically with
  $\Rgal$ and $|z|$, in qualitative agreement with the findings of 
  \cite{Nidever2014} and \cite{Hayden2015}.  See Fig.~\ref{fig7}.
%%%
\item The $\feh$ and $\oh$ distributions of stars near the Galactic plane
  ($|z|<0.5\kpc$) change shape, from negatively skewed at small $\Rgal$
  to roughly symmetric in the solar neighborhood to positively skewed in
  the outer Galaxy, in agreement with the findings of \cite{Hayden2015}
  and our new measurements based on APOGEE DR16.  The influence of radial
  migration on MDF shape agrees with the simplified model presented
  by \cite{Hayden2015} and with the numerical simulation results of
  \cite{Loebman2016}.  See Figs.~\ref{fig10} and~\ref{fig11}.
%%%
\item Moving up from the midplane, the $\feh$ and $\oh$ MDFs become more
  symmetric and less dependent on $\Rgal$, again in agreement with the
  observational findings of \cite{Hayden2015} and the simulation results
  of \cite{Loebman2016}.  However, the high-$|z|$ MDFs are not a perfect
  match to the APOGEE data, especially at $|z|=0.5-1\kpc$ where they 
  remain too skewed and too $\Rgal$-dependent.  See Figs.~\ref{fig10}
  and~\ref{fig11}.
%%%
\item The distributions of $\ofe$ in bins of $\feh$ are broad, and their
  skewness and width change with $\Rgal$ and $|z|$ in qualitative 
  agreement with the APOGEE-based measurements of \cite{Vincenzo2021}.
  However, the model $\ofe$ distributions at sub-solar $\feh$ do not
  reproduce the pronounced bimodality found by \cite{Vincenzo2021}, and
  the centroid of the model $\ofe$ distribution at super-solar $\feh$
  shifts upwards with increasing $|z|$, a trend not seen in the data.
  See Fig.~\ref{fig12}.
%%%
\item The trend of median stellar age in bins of $\ofe$ agrees with the
  measurements of \cite{Feuillet2019} in the solar neighborhood.  The 
  width of the log(age) distribution is narrow at high $\ofe$ and
  broad near solar $\ofe$, again in agreement with the data.  The model
  predicts a median $\ofe$-age relation that is nearly constant over the
  range $\Rgal=5-13\kpc$ and $|z|=0-2\kpc$, but \cite{Feuillet2019} find
  a $\sim 20\%$ reduction in the median age of high-$\ofe$ stars at high $|z|$.
  See Figs.~\ref{fig13} and~\ref{fig14}.
%%%
\item While most stars with $\ofe \geq 0.1$ are old, the model predicts a
  significant population of young and intermediate-age $\alpha$-rich stars.
  These stars form in the outer Galaxy ($\Rgal > 10\kpc$) at times
  when the local SNIa rate, and thus iron enrichment, has fluctuated to low
  values because stellar populations have migrated away before most of their
  SNIa have time to explode (Fig.~\ref{fig8}).  This mechanism, which is
  only realized because we track SNIa enrichment through annuli as 
  populations migrate (\S\ref{sec:2p2}), offers a novel explanation for the
  existence of young and intermediate-age $\alpha$-rich stars seen in
  APOGEE \citep{Chiappini2015,Martig2015,Warfield2021}.  
  See Figs.~\ref{fig13} and~\ref{fig14}.
%%%
\item In the solar neighbourhood, the predicted distribution of stellar age
  at solar $\feh$ or $\oh$ is broad, and the trend of median age with 
  metallicity is non-monotonic, with both sub-solar and super-solar metallicity
  stars being older on average than solar metallicity stars.  These 
  predictions agree with the observational results of \cite{Feuillet2019},
  though the age scatter is larger than \cite{Feuillet2019} infer, and
  the agreement of median trends is better for $\oh$ than for $\feh$.
  The old population at super-solar metallicity has migrated from the 
  inner Galaxy, as suggested by \cite{Feuillet2018,Feuillet2019}, and 
  migration produces a non-monotonic age-metallicity relation in the
  solar neighborhood even if the SFH is constant throughout the disc.
  The agreement between predicted and observed age-$\feh$ relations is
  noticeably worse at $\Rgal=5-7\kpc$ and somewhat worse at $\Rgal=11-13\kpc$.
  See Figs.~\ref{fig15}-\ref{fig17}.
%%%
\item The models with late bursts of star formation, either throughout the
  disc or at $\Rgal > 6\kpc$ only (see Fig.~\ref{fig4}), achieve better
  agreement with the \cite{Feuillet2019} age-$\feh$ relation over a range
  of $\Rgal$.  In particular, these models better reproduce the young median
  ages of solar metallicity stars and the C-shaped form of the observed
  relation.  However, they predict a $\sim 0.1$-dex uptick of $\ofe$ at
  ages of $\sim 2\Gyr$ that is not seen in the data.  The SFH of these 
  models is empirically motivated \citep{Isern2019,Mor2019}, and we do not
  know if there is some variation of our implementation that would
  preserve their improved agreement with age-$\feh$ while mitigating
  their mismatch to age-$\ofe$.  For the other measures listed above, the
  predictions of these models are qualitatively similar to those of 
  the inside-out model.  See Figs.~\ref{fig14} and~\ref{fig17}.
\end{enumerate}

All of these predictions are affected by radial migration, and those involving
vertical trends also inherit the simulation's predicted correlations between
final $|z|$ and the age, birth radius, and final radius of a stellar population.
We regard the overall level of agreement with many distinctive features of the
Milky Way disc's abundance structure as a signficant success of the models.
However, at least within the range explored here, it appears that models
with smooth star formation, accretion, and outflow histories coupled to
stellar migration are not able to explain the pronounced bimodality of 
the observed $\afe$ distribution.  As discussed by \cite{Vincenzo2021},
we expect that this problem is generic: a one-zone model with a smooth
SFH produces an $\afe$ distribution that peaks at low values, so it is
difficult to create a superposition of such models that has a bimodal 
distribution.

The most widely explored solution to this problem involves a two-phase SFH,
with gas accretion resetting the ISM to low metallicity in between.
Versions of this scenario arise in two-infall GCE models
\citep{Chiappini1997,Spintoni2018,Khoperskov2021} and in cosmological
simulations that give rise to bimodal $\afe$
\citep{Mackereth2017,Grand2018,Buck2020}.  
An alternative scenario proposed by \cite{Clarke2019}, motivated by
hydrodynamic simulations, attributes the low-$\alpha$ sequence to an
evolutionary track with low star formation efficiency and the high-$\alpha$
sequence to clumpy bursts of star formation that self-enrich with
$\alpha$ elements.  In a third, possibly fanciful scenario proposed by
\cite{Weinberg2017}, increased outflow efficiency at late times causes
the low-$\alpha$ population to evolve ``backwards'' to lower $\feh$, after
formation of the high-$\alpha$ sequence.  Radial migration is likely
to reshape the predictions of any of these scenarios, even if it does not
fully explain bimodality on its own.  These scenarios can be easily realized
within our modeling framework by changing SFH, star formation efficiency,
and outflow parameterizations.  We intend to explore the in future work,
seeking observable signatures that can distinguish these alternative 
explanations for one of the most striking features of the disc abundance
distribution.

The computational speed of our hybrid chemical evolution methodology makes it
a valuable complement to calculating chemical evolution within full 
hydrodynamic cosmological simulations 
(e.g., \citealt{Mackereth2017,Grand2018,Naiman2018,Buck2020,Vincenzo2020}).
% Others to cite here?
For a given cosmological simulation, we can consider many different choices
of yields and chemical evolution parameters, varying them individually to
isolate physical effects and exploring parameter space to identify good
fits, degeneracies, and persistent discrepancies with data.  There are 
many obvious directions to go in extending this approach.  One is to apply
it to additional cosmological simulations to understand the impact of 
different dynamical histories, and to add the ability to model stellar
populations accreted from satellites.  A second is to consider additional
elements that probe different nucleosynthetic pathways; many of these are
already incorporated in \vice, and it is easy to add other elements and sources.
A third is to include treatment of radial gas flows and fountains, both
of which have been explored in more idealized GCE models
(e.g., \citealt{Lacey1985,Bilitewski2012,Kubryk2015a,Kubryk2015b,
Spitoni2013,Pezzuli2016,Sharda2021}).  More ambitious is to implement
treatments of stochastic enrichment and incomplete ISM mixing
(e.g., \citealt{Montes2016,Krumholz2018b,Beniamini2020}), which have so 
far been little explored in the context of Milky Way disc evolution
but which are likely important in understanding the detailed correlations
of elemental abundances \citep{Ting2021}.  As multi-element spectroscopic
surveys grow even further in scope and precision, efficient and flexible
theoretical models will be essential for extracting the lessons they
have to teach about the origin of elements and the history of the Milky Way.
